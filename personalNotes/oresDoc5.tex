\documentclass[12pt,a4paper]{article}
\usepackage[ngerman]{babel}
\usepackage[utf8]{inputenc}
\usepackage[unicode=true,bookmarks=false,bookmarksopen=true]{hyperref}

\usepackage{xcolor}
\usepackage{graphicx}
\usepackage{tikz}

\usepackage{listings}

\def\checkmark{\tikz\fill[scale=0.4](0,.35) -- (.25,0) -- (1,.7) -- (.25,.15) -- cycle;}

\definecolor{pGreen}{rgb}{0.44, 0.71, 0}
\definecolor{nRed}{rgb}{0.74, 0, 0}

\title{ORES Custom Documentation V}
%\author{Tom Gülenman}
\date{}
\begin{document}
\maketitle
\textit{Disclaimer: No guarantee for the correctness of information / explanations / sources is given.}\\
%
\section*{Goals}
\begin{enumerate}
\item Metrics list:
\begin{itemize}
\item Add examples \checkmark
\item Correct the descriptions of counts and rates \checkmark
\item Improve descriptions of roc\_auc and pr\_auc \checkmark
\item Add the new standalone version to the repo \checkmark
\end{itemize}
\item Research what form of revision data is needed (for existing visualizations, but also in general) $\rightarrow$ also check RecentChanges and new Filters and what API calls look like in that context
\item 
\begin{itemize}
\item Watch \href{https://www.dataschool.io/roc-curves-and-auc-explained}{``ROC curves and Area Under the Curve explained''} and add insights to the \textbf{roc\_auc} in the list \checkmark
\item and think about what parameters could be used in which ways to filter the output of the current UI (currently: X inputs $\rightarrow$ X outputs)
\item Also check out: \href{https://www.kaggle.com/general/7517#post41179}{Precision-Recall AUC vs ROC AUC discussion} \checkmark
\end{itemize}
\item Read \href{https://doi.org/10.1145/3185517}{A Review of User Interface Design for Interactive Machine Learning}
\item Think about what could be the goal of this thesis
\end{enumerate}
%
%
%
\newpage
\section{Crucial metrics: \textbf{damaging}-model}
Examined metrics:
\begin{itemize}
\item !f1
\item !precision
\item !recall
\item accuracy
\item counts
\item f1
\item filter\_rate
\item fpr
\item match\_rate
\item pr\_auc
\item precision
\item rates
\item recall
\item roc\_auc
\end{itemize}
For each metric (if possible) there will be:
\begin{enumerate}
\item The formula based on the \textbf{confusion matrix}
\item A definition
\item An intuitive explanation with an example
\item Its meaning based on the \textbf{loan threshold} representation by Google (\href{https://research.google.com/bigpicture/attacking-discrimination-in-ml/}{Link})
\item Additional information (if necessary)
\end{enumerate}
\subsection*{Explanations: References}
\begin{itemize}
\item Confusion Matrix 
\begin{description}
\item \includegraphics[scale=0.3]{resources/3/confusionMatrix}
\item Abbreviations: \textbf{\textcolor{pGreen}{TP}}, \textbf{\textcolor{nRed}{FP}}, \textbf{\textcolor{nRed}{FN}} and \textbf{\textcolor{pGreen}{TN}}.
\end{description}
\item Loan Threshold
\begin{description}
\item \includegraphics[scale=0.57]{resources/4/loanML4}
\end{description}
\end{itemize}
%
\subsection*{The example scenario}
To ease the understanding, let's stick to the following scenario and refer to it for each metric:
\begin{itemize}
\item 100 people represent our total population
\item 35\% of our population is infected with disease X
\begin{description}
\item That leaves us with the following \textbf{labels}: \textbf{35} \colorbox{pGreen}{\textbf{positives}} and \textbf{65} \colorbox{nRed}{\textbf{negatives}} (being tested for disease X;)
\item \includegraphics[scale=0.5]{resources/5/exampleBase}
\end{description}
\item A classifier is now supposed to classify every person of our population (based on their visible symptoms for example). This is the \textbf{prediction}:
\begin{itemize}
\item If our algorithm says a person is infected, that person will be classified as a \colorbox{pGreen}{\textbf{positive}}, and marked with radioactive symbol:
\begin{description}
\item \includegraphics[scale=0.3]{resources/5/radioactive}
\end{description}
\item If the prediction results in a \colorbox{nRed}{\textbf{negative}}, the person will be marked with a sun symbol:
\begin{description}
\item \includegraphics[scale=0.3]{resources/5/sun}
\end{description}
\end{itemize}
\item The classifier may predict that:
\begin{itemize}
\item out of the 35 \colorbox{pGreen}{infected people}, \textbf{30} are infected (those 30 are what we call \textbf{true positives}) and \textbf{5} are not (those 5 are \textbf{false negatives}) 
\item out of the 65 \colorbox{nRed}{non infected people}, \textbf{10} are infected (\textbf{false positives}) and \textbf{55} are not (\textbf{true negatives})
\begin{description}
\item \includegraphics[scale=0.5]{resources/5/examplePredicted}
\item \includegraphics[scale=0.5]{resources/5/exampleTP} \textbf{true positive}: infected and correctly predicted (\textbf{30})
\item \includegraphics[scale=0.5]{resources/5/exampleFN} \textbf{false negative}: infected and incorrectly predicted (\textbf{5})
\item \includegraphics[scale=0.5]{resources/5/exampleTN} \textbf{true negative}: not infected and correctly predicted (\textbf{55})
\item \includegraphics[scale=0.5]{resources/5/exampleFP} \textbf{false positive}: not infected and incorrectly predicted (\textbf{10})
\end{description}
\end{itemize}
\end{itemize}

\paragraph{Let's get started.}
\subsection{recall}
\begin{enumerate}
\item $\frac{\texttt{TP}}{\texttt{TP} + \texttt{FN}}$
\item Recall ($\equiv$ true positive rate $\equiv$ ``sensitivity'') is the ability of a model to find \textbf{all} relevant cases within the dataset.
\item Now, in our example scenario, the relevant cases are the infected people. We absolutely want to identify those: \includegraphics[scale=0.3]{resources/5/exampleP}. The ability of the model to identify those depends on how many will be \textbf{correctly} predicted to be infected: \includegraphics[scale=0.3]{resources/5/exampleTP}.
\begin{description}
\item In other words, we are looking for the ratio of correctly predicted to be infected people to all infected people.
\item That leads to $\frac{\sum \includegraphics[scale=0.3]{resources/5/exampleTP}}{\sum \includegraphics[scale=0.3]{resources/5/exampleP}}$, with $\includegraphics[scale=0.3]{resources/5/exampleP} = \includegraphics[scale=0.3]{resources/5/exampleTP} + \includegraphics[scale=0.3]{resources/5/exampleFN}$, which is equivalent to the formula in \textbf{1.}, if you replace the symbols with their confusion matrix counterpart according to the legend in \textbf{The example scenario}.
\item In terms of numbers for our example that would be $\frac{30}{30+5} \approx 0.86$
\end{description}
\item $\frac{\includegraphics[scale=0.6]{resources/4/loanTP}}{\includegraphics[scale=0.6]{resources/4/loanTP+FN}}$
\end{enumerate}
%
\subsection{precision}
\begin{enumerate}
\item $\frac{\texttt{TP}}{\texttt{TP} + \texttt{FP}}$
\item Ability of the model to find \textbf{only} relevant cases within the dataset
\item Again, we take a look at the relevant cases, the infected people: \includegraphics[scale=0.3]{resources/5/exampleP}. This time around though, we are \textbf{not} interested in the ratio of correctly predicted to be infected people to \textbf{all} infected people. Instead we want to know how good the model is at only predicting those to be infected, that actually are. Therefore, we want the ratio of all \includegraphics[scale=0.3]{resources/5/exampleTP} to all those predicted to be infected: $\includegraphics[scale=0.3]{resources/5/exampleTP} + \includegraphics[scale=0.3]{resources/5/exampleFP}$
\begin{description}
\item $= \frac{\sum \includegraphics[scale=0.3]{resources/5/exampleTP}}{\sum \includegraphics[scale=0.3]{resources/5/exampleTP} + \sum \includegraphics[scale=0.3]{resources/5/exampleFP}} = \frac{30}{30+10} = 0.75$
\end{description}
\item $\frac{\includegraphics[scale=0.6]{resources/4/loanTP}}{\includegraphics[scale=0.6]{resources/4/loanTP+FP}}$
\end{enumerate}
%
\subsection{f1}
\begin{enumerate}
\item -
\item f1-Score, the harmonic mean of recall and precision, a metric from \textbf{0} (worst) to \textbf{1} (best), used to evaluate the accuracy of a model by taking into account recall and precision: $=2*\frac{\texttt{precision} * \texttt{recall}}{\texttt{precision}+\texttt{recall}}$
\item For our example model, that would result in $=2*\frac{0.75 * \frac{30}{35}}{0.75+\frac{30}{35}} = 0.8$
\item -
\item \underline{Additional information}: Compared to the simple average (of recall and precision), the harmonic mean punishes extreme values (e.g. precision 1.0 and recall 0.0 $\rightarrow$ average = 0.5, but f1 $= 0$)
\end{enumerate}
%
\subsection{fpr}
\begin{enumerate}
\item $\frac{\texttt{FP}}{\texttt{FP} + \texttt{TN}}$
\item The false positive rate is the probability of a false alarm.
\item In our example, a false alarm would obviously be labeling someone as infected, who isn't: \includegraphics[scale=0.3]{resources/5/exampleFP}. Now we just have to ask ourselves what portion of those, that, if they \textbf{were} incorrectly predicted as infected (because they \textbf{are not} infected: \includegraphics[scale=0.3]{resources/5/exampleN}), \textbf{are} incorrectly predicted as infected? Hence, we are looking for the ratio of \includegraphics[scale=0.3]{resources/5/exampleFP} to all non infected people: $\includegraphics[scale=0.3]{resources/5/exampleFP} + \includegraphics[scale=0.3]{resources/5/exampleTN}$.
\begin{description}
\item $= \frac{\sum \includegraphics[scale=0.3]{resources/5/exampleFP}}{\sum \includegraphics[scale=0.3]{resources/5/exampleFP} + \sum \includegraphics[scale=0.3]{resources/5/exampleTN}} = \frac{10}{10+55} \approx 0.15$
\end{description}
\item $\frac{\includegraphics[scale=0.6]{resources/4/loanFP}}{\includegraphics[scale=0.6]{resources/4/loanTN+FP}}$
\end{enumerate}
\subsection{roc\_auc}
\begin{enumerate}
\item -
\item The \textbf{area under} the \textbf{curve} of the \textbf{ROC}-curve, a measure between 0.5 (worthless) and 1.0 (perfect: getting no FPs), rates the ability of a model to achieve a blend of recall and precision.
\item In our example, we haven't used the notion of threshold yet. For classifying people as infected or not, the classifier will evaluate multiple criteria and calculate the probability that a patient is infected. Many binary classifiers have the threshold at 0.5, meaning that, if the probability of a true outcome is higher than 50\%, it is classified as a positive; or in our case as an infected patient. Depending on the situation however, it can be useful to move that threshold.
\begin{description}
\item  The receiver operating characteristic (ROC) curve is used to visualize the performance of a classifier
\item The ROC curve plots the TPR versus FPR as a function of the model’s threshold for classifying a positive.
\item \includegraphics[scale=0.5]{resources/3/ROCcurve}
\item Increasing the threshold $\rightarrow$ moving up a curve ($\equiv$ model) to the top right corner, where all data is predicted as positive (threshold = 1.0) and vice versa
\item Our best case is a curve hugging the top left corner, because that means a low \textbf{fpr} and high \textbf{tpr}, which, again, means a lot of positives and few negatives on the right of the thresholds, looking at class curves like the \textbf{Loan Threshold} one.
\item Assuming that we have had a threshold of 0.5 all along to get the previous results, one point on our ROC curve would be: $(0.15,0.86)=(\texttt{fpr},\texttt{tpr})$.
\item We would now have to change the thresholds, look at the (probably) changed resulting data (new numbers in terms of positives and negatives, as well as \textbf{TP}, \textbf{FN}, \textbf{TN} and \textbf{FP}) and then plot every resulting $(\texttt{fpr},\texttt{tpr})$-point in order to plot the full ROC curve.
\item We would most certainly like to quantify this visualized performance of our binary classifier, which is why we calculate the area under the curve (\textbf{auc}) of the ROC curve.
\end{description}
\item -
\item \underline{Additional information}: We can think of AUC as representing the probability that a classifier will rank a randomly chosen positive observation higher than a randomly chosen negative observation. That's why \textbf{roc\_auc} is a useful metric even for datasets with highly unbalanced classes. (\href{https://www.youtube.com/watch?v=OAl6eAyP-yo}{Source})
\end{enumerate}
%
\subsection{pr\_auc}
\begin{enumerate}
\item -
\item Similarly to the \textbf{roc\_auc}, the area under the precision recall curve \textbf{pr\_auc} evaluates a classifiers performance. The main difference, however, is that the \textbf{pr\_auc} does not make use of \textbf{true negatives}. It is therefore favourable to use \textbf{pr\_auc} over \textbf{roc\_auc} if true negatives are unimportant to the general problem or if there are a lot more negatives than positives.
\begin{description}
\item The PR-curve plots the Precision versus the Recall:
\item \includegraphics[scale=0.7]{resources/3/PRcurve}
\item Instead of the top left corner for the ROC-curve, here, we want our curve to reach the top right corner for our classifier to be perfect.
\end{description}
\begin{description}
\item The following scenario provides a good example (\href{https://www.kaggle.com/general/7517#post41179}{Source 1} and \href{http://www.chioka.in/differences-between-roc-auc-and-pr-auc/}{Source 2}) of a case with a lot more negatives than positives and comparing ROC to PR:
\begin{itemize}
\item Out of 1 million documents, we want to find the 100 relevant ones.
\item The task is accomplished by two different algorithms:
\begin{enumerate}
\item 100 retrieved documents, 90 relevant
\item 2000 retrieved documents, 90 relevant
\end{enumerate}
\item Algorithm (a) is obviously preferable.
\item We know that ROC- and PR-curves both plot coordinates with $\texttt{tpr} = \texttt{recall}$ as one dimension. Now the question is: how do they differ in the other dimension, when plotting both algorithms?
\item In all cases $\texttt{tpr} = \texttt{recall} = 0.9$. We also have:
\begin{enumerate}
\item $\texttt{TN} = 999890$ and $\texttt{FP} = 10$
\item $\texttt{TN} = 997990$ and $\texttt{FP} = 1910$
\end{enumerate}
\item \colorbox{yellow}{ROC}:
\begin{enumerate}
\item $\texttt{fpr}= \frac{\texttt{FP}}{\texttt{FP+TN}} = \frac{10}{10+999890} = 0.00001$
\item $\texttt{fpr}= \frac{1910}{1910+997990} = 0.00191$
\begin{description}
\item Having retrieved many more documents, and therefore having many more \textbf{false positives}, algorithm (b) has a higher \textbf{fpr} than algorithm (a). 
\item The \textbf{fpr} also takes into account the vast amount of \textbf{true negatives} though, which is why the difference between the two \textbf{fpr}s is still quite small: $0.0019$.
\end{description}
\end{enumerate}
\item \colorbox{yellow}{PR}:
\begin{enumerate}
\item $\texttt{precision}= \frac{\texttt{TP}}{\texttt{TP+FP}} = \frac{90}{90+10} = 0.9$
\item $\texttt{precision}= \frac{90}{90+1910} = 0.045$
\begin{description}
\item Not accounting for \textbf{true negatives}, the \textbf{precision} is not affected by the relative imbalance.
\item We are presented a remarkable difference of $0.855$.
\end{description}
\end{enumerate}
\item To close on this topic, not the following (by Randy C): ``Obviously, those are just single points in ROC and PR space, but if these differences persist across various scoring thresholds, using ROC AUC, we'd see a very small difference between the two algorithms, whereas PR AUC would show quite a large difference.''
\end{itemize}
\end{description}  
%
\item Similarly to the \textbf{roc\_auc}, the point on the PR curve of our example for the standard threshold of 0.5 would be: $(\texttt{precision},\texttt{recall}) = (0.75.0.86)$.
\item -
\end{enumerate}
%
\subsection{accuracy}
\begin{enumerate}
\item $\frac{\texttt{TP}+\texttt{TN}}{\texttt{Total}}$
\item Accuracy measures the portion of correctly predicted data
\item In our example scenario, this is equal to asking ourselves \textit{out of all patients, what's the portion of correctly predicted cases?} The correctly predicted cases are infected patients, predicted to be infected ( \includegraphics[scale=0.3]{resources/5/exampleTP} ), and non infected patients, predicted not to be infected ( \includegraphics[scale=0.3]{resources/5/exampleTN} ). This wanted proportion results in: 
\begin{description}
\item $= \frac{\sum \includegraphics[scale=0.3]{resources/5/exampleTP} + \sum \includegraphics[scale=0.3]{resources/5/exampleTN}}{\sum \includegraphics[scale=0.3]{resources/5/exampleP} + \sum \includegraphics[scale=0.3]{resources/5/exampleN}} = \frac{30+55}{35+65} = 0.85$
\end{description}
\item $\frac{\includegraphics[scale=0.6]{resources/4/loanTP+TN}}{\includegraphics[scale=0.6]{resources/4/loanTotal2}}$
\end{enumerate}
\subsection{match\_rate}
\begin{enumerate}
\item $\frac{\texttt{TP}+\texttt{FP}}{\texttt{Total}}$
\item The match rate is the proportion of observations matched/not-matched, meaning the ratio of observations predicted to be positive to the total of observations.
\item Concerning our example, this would be equal to wanting to know what portion of the population was predicted to be infected. Those groups are: \includegraphics[scale=0.3]{resources/5/exampleTP} and \includegraphics[scale=0.3]{resources/5/exampleFP}.
\begin{description}
\item \item $= \frac{\sum \includegraphics[scale=0.3]{resources/5/exampleTP} + \sum \includegraphics[scale=0.3]{resources/5/exampleFP}}{\sum \includegraphics[scale=0.3]{resources/5/exampleP} + \sum \includegraphics[scale=0.3]{resources/5/exampleN}} = \frac{30+10}{35+65} = 0.4$
\end{description}
\item $\frac{\includegraphics[scale=0.6]{resources/4/loanTP+FP}}{\includegraphics[scale=0.6]{resources/4/loanTotal2}}$
\end{enumerate}
%
\subsection{filter\_rate}
\begin{enumerate}
\item $1-\texttt{match\_rate} = \frac{\texttt{TN}+\texttt{FN}}{\texttt{Total}}$
\item The filter rate is the proportion of observations filtered/not-filtered, meaning the ratio of observations predicted to be negative to the total of observations. This is the complement to the match rate.
\item In our example scenario, this would be equal to wanting to know what portion of the population was predicted not to be infected.
\begin{description}
\item $= \frac{\sum \includegraphics[scale=0.3]{resources/5/exampleTN} + \sum \includegraphics[scale=0.3]{resources/5/exampleFN}}{\sum \includegraphics[scale=0.3]{resources/5/exampleP} + \sum \includegraphics[scale=0.3]{resources/5/exampleN}} = \frac{55+5}{35+65} = 0.6 = 1- \texttt{match\_rate}$
\end{description}
\item $\frac{\includegraphics[scale=0.6]{resources/4/loanTN+FN}}{\includegraphics[scale=0.6]{resources/4/loanTotal}}$
\end{enumerate}
%
\subsection{counts}
\begin{enumerate}
\item \begin{itemize}
\item \textbf{labels}:
\begin{itemize}
\item false:  $\texttt{TN}+\texttt{FP}$
\item true: $\texttt{TP}+\texttt{FN}$
\end{itemize}
\item \textbf{n}: $\texttt{Total}$
\item \textbf{predictions}:
\begin{itemize}
\item false:
\begin{itemize}
\item false: $\texttt{TN}$
\item true: $\texttt{FP}$
\end{itemize}
\item true:
\begin{itemize}
\item false: $\texttt{FN}$
\item true: $\texttt{TP}$
\end{itemize}
\end{itemize}
\end{itemize}
\item \begin{itemize}
\item \textbf{labels}: The number of edits (\textit{manually}) labeled as \textit{false} and \textit{true}: these values represent the \textbf{actual} positives and negatives.
\item \textbf{n}: The sample size; total number of edits taken into account
\item \textbf{predictions}: \textcolor{blue}{edits ...}
\begin{itemize}
\item false: \textcolor{nRed}{... actually being false ...}
\begin{itemize}
\item false: \textcolor{nRed}{... and predicted to be false}
\item true: \textcolor{nRed}{... but predicted to be true}
\end{itemize}
\item true: \textcolor{pGreen}{... actually being true ...}
\begin{itemize}
\item false: \textcolor{pGreen}{... but predicted to be false}
\item true: \textcolor{pGreen}{... and predicted to be true}
\end{itemize}
\end{itemize}
\end{itemize}
\item Concerning our example:
\begin{itemize}
\item \textbf{labels}:
\begin{itemize}
\item false (\textcolor{nRed}{non infected people}): $\includegraphics[scale=0.3]{resources/5/exampleTN}+\includegraphics[scale=0.3]{resources/5/exampleFP}=55+10=65$ 
\item true (\textcolor{pGreen}{infected people}): $\includegraphics[scale=0.3]{resources/5/exampleTP}+\includegraphics[scale=0.3]{resources/5/exampleFN}= 30+5=35$
\end{itemize}
\item \textbf{n} (Total): $100$
\item \textbf{predictions}:
\begin{itemize}
\item false (\textcolor{nRed}{non infected people ...})
\begin{itemize}
\item false (\textcolor{nRed}{... and predicted not to be infected}): $\includegraphics[scale=0.3]{resources/5/exampleTN} = 55$
\item true (\textcolor{nRed}{... but predicted to be infected}): $\includegraphics[scale=0.3]{resources/5/exampleFP} = 10$
\end{itemize}
\item true (\textcolor{pGreen}{infected people ...})
\begin{itemize}
\item false (\textcolor{pGreen}{... predicted not to be infected}): $\includegraphics[scale=0.3]{resources/5/exampleFN} = 5$
\item true (\textcolor{pGreen}{... predicted to be infected}): $\includegraphics[scale=0.3]{resources/5/exampleTP} = 30$
\end{itemize}
\end{itemize}
\end{itemize}
\item -
\item \underline{Additional information}:
\begin{description}
\item When calling the enwiki damaging model for example (\href{https://ores.wikimedia.org/v3/scores/enwiki/?models=damaging&model_info=statistics}{Link}), you will get the following output for counts:
\item \includegraphics[scale=0.7]{resources/4/enwikiDamagingCounts}
\item $\Rightarrow$ e.g. out of 18677 edits that were labeled as \textit{false}, 719 were false positives
\end{description}
\end{enumerate}
%
\subsection{rates}
\begin{enumerate}
\item \begin{itemize}
\item false: $\frac{\texttt{counts\_labels\_false}}{\texttt{counts\_n}}$
\item true: $\frac{\texttt{counts\_labels\_true}}{\texttt{counts\_n}}$
\end{itemize}
\item The rates simply give us the proportion of edits labeled as false or true to the total number of edits taken into account.
\item For the example scenario: 
\begin{itemize}
\item false (proportion of infected people to the total number of people tested): $\frac{\sum \includegraphics[scale=0.3]{resources/5/exampleN}}{100} = \frac{65}{100} = 0.65$
\item true (proportion of infected people to the total number of people tested): $\frac{\sum \includegraphics[scale=0.3]{resources/5/exampleP}}{100} = \frac{35}{100} = 0.35$
\end{itemize}
\item -
\item \underline{Additional information}:
\begin{description}
\item Calling the API the same way as for \textbf{counts} (\href{https://ores.wikimedia.org/v3/scores/enwiki/?models=damaging&model_info=statistics}{Link}), we get:
\item \includegraphics[scale=0.7]{resources/4/enwikiDamagingRates}
\item The number of edits taken into account for \textbf{sample} equals the \textbf{n} from \textbf{counts} $= 19428$.
\item Now we see that, also looking at the output under \textbf{counts}, \textit{rates\_sample\_false}: $0.961 = \frac{18677}{19428}$.
\item Note that we are shown results for ``population'' and ``sample''. There is a significant number of bot edits and edits that don't need reviewing (admins, autopatrolled users). The \textbf{sample} of edits does not contain any of those.
\end{description}
\end{enumerate}
%
\subsection{!$<$metric$>$}
\begin{itemize}
\item Any $<$metric$>$ with an exclamation mark is the same metric for the negative class
\item e.g. recall $= \frac{\texttt{TP}}{\texttt{TP} + \texttt{FN}} \Rightarrow$ \textbf{!}recall  $= \frac{\texttt{TN}}{\texttt{TN} + \texttt{FP}}$
\item Example usage: find all items that are not ``E'' class  (itemquality model) $\rightarrow$ look at \textbf{!}recall for ``E'' class.
\end{itemize}
\subsubsection{Existing !$<$metric$>$s}
\begin{itemize}
\item !f1
\item !precision
\item !recall
\end{itemize}
%
%
%
\section{Revision data, RecentChanges and newFilters}
%TODO
%
%
%
\section{pr\_auc, roc\_auc and filtering the ORES UI output}
\subsection{pr\_auc and roc\_auc}
Insights from the linked video and discussion under \textbf{Goals} have been added to the metrics list.
\subsection{What parameters could be used to filter the output of the current UI?}
%TODO

\section{\href{https://doi.org/10.1145/3185517}{A Review of User Interface Design for Interactive Machine Learning} - Notes}
\begin{itemize}
\item Interactive Machine Learning (\textbf{IML}) complements human percept. + int.: computat. speed + power
\item Interactive process: input from user without needing knowledge of ML
\begin{description}
\item $\Rightarrow$ UI Design is fundamental!
\item ... but no consolidated principles on implement. of such a UI
\end{description}
\item In this paper: proposit. of model of generalised IML system and solut. principles for effective IML interf.
\end{itemize}
\subsection{Introduction}
\begin{itemize}
\item ML: application as development tool only for experts
\item IML: training process as HCI task $\rightarrow$ more accessible
\begin{description}
\item for example: human input in example selection, creation and labelling
\end{description}
\item Note: expert may still be required to underlying alg.
\item Ideal case: non expert constructs own learned concepts by creating or collecting training data according to their need
\begin{description}
\item In practice: major challenge for ML practitioner (=expert) and UI designer!
\item $\Rightarrow$ This paper: focuses on interface design challenge
\end{description}
\item IML: allows for application of domain knowledge (domain experts can now train models)!
\item IML workflow is co-adaptive: user and target model directly influence each other's behaviour
\begin{description}
\item Again, this is done over the interface: it's a dialogue between human and machine
\end{description}
\item Four key challenges for designing and IML interface:
\begin{enumerate}
\item Users are imprecise and inconsistent. ``Unless users perceive their own deficiencies, this failure [poor model resulting from poor training by user] is attributed to the system.''
\item Uncertainty relating user input and user intent. E.g. user doesn't assign an example to a concept does not imply a counter example
\item Model is not your conventional information structure: ``ML model evolves in response to user input but not necessarily in a way that is perceived as intuitive or predictable by the user''
\item Training is open ended.
\end{enumerate}
\end{itemize}
\subsubsection{Guidance for the designer}
Structural and behavioural breakdown of IML system will be presented.
\begin{description}
\item Help designers think about the architecture and support discussion of design considerations with clear terminology
\item Also help with how to present information to users and what interactions to promote
\end{description}
\begin{itemize}
\item Structural: Constituent components (user, interface, data, model)
\begin{description}
\item Interface will be divided into four elements itself.
\end{description}
\item Behavioural: Seperate process of interactively building model into sub-tasks
\begin{description}
\item Tasks may overlap / be performed iteratively
\end{description}
\end{itemize}
%TODO ``The six solution principles identified and described in detail in Section 6''
\subsubsection{Outline}
\begin{itemize}
\item Section 2: Detailed definition of IML
\item Section 3: Overview of efforts at making ML more accessible
\item Section 4: Key interface elements of generalised IML stystems
\item Section 5: IML process: an abstracted description
\item Section 6: Identifying common solution principles
\item Section 7: Proposition of further research
\item Section 8: Summary
\end{itemize}
\subsection{Context and definitions (``Section 2'')}
\subsubsection{IML}
\begin{itemize}
\item \textit{Def.:} Interact. paradigm where user (+-group) iteratively builds and refines mathematical model to describe concept through iterative cycles of input and review
\item Differences to traditional ML:
\begin{description}
\item Human int. is applied through iterative teaching in tight loops
\item $\Rightarrow$ change in the model at each iterat. is relatively small
\item But also: less pre-selection of training data needed
\item User is principle driver of interaction
\end{description}
\item Examples of IML: Crayons and ReGroup. IML is esp. beneficial when objectives of user ``and/or data labels cannot be obtained a priori''
\item %TODO weiter auf seite 5
\end{itemize}

\section{Possible goal of the thesis}
%TODO

%
%
%
%
%
\section*{Questions}
\begin{itemize}
%
\item \colorbox{yellow}{Q:} Should I ask Aaron how he would like us to work together?
\begin{description}
\item \colorbox{orange}{A:} 
\end{description}
%
\item \colorbox{yellow}{Q:} In what situations exactly do we want to optimize the threshold in the context of user centered threshold optimization?
\begin{description}
\item \colorbox{orange}{A:} 
\end{description}
%
\end{itemize}
\end{document}